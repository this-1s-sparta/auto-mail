\documentclass[a4paper,12pt]{article}
\usepackage{amsmath}
\usepackage[english, greek]{babel}
\usepackage{graphicx}
\usepackage{hyperref} % Για σωστή διαχείριση URL

\title{Οδηγίες Αποστολής Βαθμολογιών}
\date{\today}

\begin{document}

\maketitle

\selectlanguage{greek}
\section{Προετοιμασία}
\begin{itemize}
    \item Τα αρχεία \selectlanguage{english} Excel \selectlanguage{greek} πρέπει να είναι \textbf{κλειστά}.
    \item \selectlanguage{english}  dont\_touch.xlsx  \selectlanguage{greek} θα περιέχει ΑΕΜ, Ονοματεπώνυμο,\selectlanguage{english} Email\selectlanguage{greek}.
    \item \selectlanguage{english}"grades.xlsx" \selectlanguage{greek} θα περιέχει ΑΕΜ και βαθμολογίες.
    \item \textbf{Μην αλλάζετε τον φάκελο και μη διαγράφετε τις πρώτες γραμμές στα  \selectlanguage{english} Excel \selectlanguage{greek}}.
\end{itemize}

\selectlanguage{greek}
\section{Αποστολή Βαθμών}
\subsection{Από \selectlanguage{english} Gmail \selectlanguage{greek} σε \selectlanguage{english} csd.auth \selectlanguage{greek}}
\begin{enumerate}
    \item H γραμμη 17 πρέπει να είναι : \selectlanguage{english} SMTP\_SERVER = "smtp.gmail.com"\selectlanguage{greek}
    \item Ενεργοποιήστε \selectlanguage{english} 2-step authentication \selectlanguage{greek} στο \selectlanguage{english} Gmail. \selectlanguage{greek}
    \item Δημιουργήστε κωδικό εφαρμογής στο \href{https://myaccount.google.com/apppasswords}{\selectlanguage{english} Google App Passwords \selectlanguage{greek}}.
    \item Εκτελέστε το πρόγραμμα, εισάγετε  \selectlanguage{english}email \selectlanguage{greek} και τον 16-ψήφιο κωδικό.
    \item Εισάγετε όνομα μαθήματος και τρέξτε το πρόγραμμα.
\end{enumerate}

\subsection{Από \selectlanguage{english} csd.auth \selectlanguage{greek} σε \selectlanguage{english} csd.auth \selectlanguage{greek}}
\begin{enumerate}
    \item Αλλάξτε στη γραμμή 17 απο  \selectlanguage{english}"smtp.gmail.com"\selectlanguage{greek} σε \selectlanguage{english}"mail.auth.gr". \selectlanguage{greek}
    \item Εισάγετε \selectlanguage{english} email \selectlanguage{greek}, κωδικό και όνομα μαθήματος.
\end{enumerate}

\section{Ειδικός Λογαριασμός}
\begin{itemize}
   \itemΗ γραμμη 17 πρέπει να είναι :\selectlanguage{english} SMTP\_SERVER = "mail.auth.gr" \selectlanguage{greek}
    \item Αν δοθεί μια ξεχωρηστή διευθύνση \selectlanguage{english}mail\selectlanguage{greek} για αυτο το σκοπό τοτε μπορεί να χρησιμοποιηθεί ΜΟΝΟ ΜΙΑ για ολούς όσους θα το χρησιμοποιήσουν.
    \item Σε αυτό το σενάριο ο κώδικας θα έχει πάνω αυτόματα τη διεύθηνση και τον κωδικό και θα γίνει αλλαγή στη γραμμη 17 απο \selectlanguage{english}"smtp.gmail.com" \selectlanguage{greek}σε \selectlanguage{english}"mail.auth.gr".\selectlanguage{greek}
    \item Ο χρήστης θα αλλάζει μόνο το ΟΝΟΜΑ ΜΑΘΗΜΑΤΟΣ.
	
\end{itemize}

\subsection{Πλεονεκτήματα}
\begin{itemize}
    \item \textbf{Ασφάλεια}: Λιγότερη έκθεση δεδομένων.
    \item \textbf{Ταχύτητα}: Πιο γρήγορη διαδικασία.
\end{itemize}

\end{document}
